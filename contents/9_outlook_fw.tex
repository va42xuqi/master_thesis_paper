\chapter{Future Work}
\label{chapt:outlook}
This thesis has demonstrated the effectiveness of using hybrid models with LMU encoders in forecasting human motion within dynamic sports environments. However, several areas for further research and improvement remain that could enhance the performance and applicability of these models:

\begin{itemize}
    \item \textbf{Deeper Hyperparameter Tuning:} While the models have shown strong performance, there is significant potential for improvement through more extensive hyperparameter tuning. Future work could involve a systematic exploration of the hyperparameters, such as the number of layers, attention heads, and the size of the LMU memory. Such deeper tuning could lead to models that are better tailored to specific sports or scenarios, unlocking new levels of performance.

    \item \textbf{Optimization for Real-Time Applications:} Another key area for future research is optimizing these models for real-time applications. The current models, while accurate, may require further optimization to meet the computational constraints of real-time sports analytics systems. This could involve reducing inference time, minimizing computational overhead, and exploring techniques to streamline the models without sacrificing accuracy. Achieving real-time performance would make these models more practical for use in live game situations, where timely predictions are critical.

    \item \textbf{Exploring Different Attention Mechanisms:} The use of attention mechanisms has been central to the success of the Transformer-based models in this thesis. However, there is room to explore different types of attention mechanisms or variations of the current approach. Future work could investigate how modifications to the attention mechanism might impact model performance, particularly in handling the complexities of dynamic sports environments. This exploration could lead to even more efficient and effective models.

    \item \textbf{Continued Generalization Across Different Sports:} The ability of the models to generalize across different sports has been a significant focus of this work. Continuing to test and adapt these models for various sports, as well as fine-tuning them for different game conditions and player behaviors, will be important. Ensuring that the models are robust across a wide range of scenarios will enhance their utility and applicability in real-world sports analytics.

    \item \textbf{Integration of Contextual Information:} Future work could also explore the integration of additional contextual information into the models. This could include environmental factors, such as weather conditions or crowd dynamics, which may influence player behavior. Incorporating this context could enhance the model's predictive accuracy and provide more nuanced insights for sports analytics.

    \item \textbf{Transfer Learning for Cross-Domain Applications:} The transfer learning approaches explored in this thesis have shown promise in cross-sport applications. Future research could extend this to explore cross-domain transfer learning, where models trained on sports data could be adapted to other fields, such as pedestrian movement forecasting or robotics. This would not only test the adaptability of the models but also potentially broaden their applicability to new areas.
\end{itemize}

By focusing on these areas, future research could further refine and expand the capabilities of the models developed in this thesis, ensuring that they continue to meet the evolving needs of sports analytics and beyond.
