
\section{Motivation and Problem Statement}
\label{sec:motivation}

Forecasting methods for stock markets, electricity prices, and weather conditions are well-established and extensively researched \cite{rapach2020, ribeiro2021, mills2019}. In contrast, forecasting human motion in sports remains relatively underexplored, yet it is crucial for reducing system delays, minimizing injuries, gaining tactical advantages \cite{advantages1}, and enhancing game analysis at various stages \cite{postgame}. For example, effective forecasting enables coaches to make informed decisions, such as substituting players to prevent injuries due to irregular motion. The growing demand for forecasting methods is driven by advancements in communication platforms, such as 6G, and the Internet of Things. However, forecasting human motion, particularly positions, is challenging due to its inherently indeterministic nature, especially in fast-paced sports environments such as NBA basketball.

Experts are investigating models that can accurately, reliably, and promptly predict human motion in dynamic settings. Traditional Bayesian methods, such as Kalman filters (KF) and particle filters (PF) \cite{diss_tobi}, have been employed, but these linear models often fall short because they do not retain the motion history necessary for forecasting beyond a single time step. In sports, where players exhibit rapid changes in speed and acceleration while influencing each other tactically, KF and PF frequently prove inadequate. This limitation has led to the exploration of AI approaches capable of uncovering nonlinear patterns in complex motion that traditional methods might miss. AI can observe, interpret, and learn movement patterns and their interactions, providing more accurate forecasts for complex and dynamic motions.

The Transformer model, introduced by Vaswani et al. \cite{transformer}, leverages the self-attention mechanism to enhance memory of past time steps, outperforming traditional recurrent neural networks (RNNs) such as LSTMs \cite{lstm} and LMUs \cite{lmu} in terms of accuracy and robustness. Transformers combine the advantages of convolutional networks (optimal feature extraction) and RNNs (memory) and are resilient against large, complex input data spaces. While examples of Transformer-based models exist, such as ChatGPT for natural language processing \cite{gpt} and TimeGPT for time series data processing \cite{timegpt}, a Transformer model specifically designed for forecasting human motion in sports, including multivariate and multi-task scenarios, has yet to be developed.

Given these challenges, predicting human motion in dynamic environments, such as \gls{nba} basketball, \gls{dfl} soccer, and pedestrian scenarios, presents a significant challenge. Accurate motion forecasting is critical for various applications, including enhancing game strategy, reducing injuries, and improving real-time analysis. Existing methods, including traditional Bayesian approaches and various AI models, often struggle to deliver reliable predictions under these complex conditions.

The primary objective of this thesis is to evaluate and identify the most effective model for forecasting the short-term movements of a single target player or individual across high dynamic domains. This research will focus on comparing multiple advanced machine learning models, with particular emphasis on Transformer models, to determine which model best captures and adapts to the unique movement patterns and behaviors of the target. By systematically assessing the performance of these models under various tasks and scenarios, this thesis aims to overcome the limitations of existing forecasting methods and provide accurate, reliable, and timely predictions in complex, dynamic environments.
