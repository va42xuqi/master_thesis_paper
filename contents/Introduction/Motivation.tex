\todo{The last thing to do: formatting}
Understanding and predicting human motion in sports is becoming increasingly important due to its implications for game strategy, injury prevention, and performance analysis. Despite significant advancements in forecasting methodologies for various domains such as finance, weather, and energy, the application of these techniques to human motion in sports is still in its nascent stages. This introduction outlines the significance of forecasting in dynamic environments, highlights the current challenges faced, and presents the scope and objectives of this thesis.
\section{Motivation}
Forecasting methods for stock markets, electricity prices, and weather conditions are well-established and extensively researched \cite{rapach2020, ribeiro2021, mills2019}. In contrast, forecasting human motion in sports remains relatively underexplored. However, this technology is crucial for reducing system delays, minimizing injuries, gaining tactical advantages \cite{advantages1}, and enhancing game analysis at various stages \cite{postgame}. For example, effective forecasting allows coaches to make informed decisions, such as substituting players to prevent injuries due to irregular motion. The growing demand for forecasting methods is driven by advancements in communication platforms, such as 6G, and the Internet of Things. Nonetheless, forecasting human motion, particularly positions, is challenging due to its inherently indeterministic nature. In fast-paced and unpredictable sports environments like NBA basketball, forecasting becomes exceedingly complex.

Experts are thus investigating models that can accurately, reliably, and promptly predict human motion in dynamic settings. Traditional Bayesian methods, such as Kalman filters (KF) and particle filters (PF) \cite{diss_tobi}, have been employed for this purpose. However, these linear models fall short as they do not retain motion history necessary for forecasting beyond a single time step. In sports, where players exhibit rapid changes in speed and acceleration and influence each other tactically, KF and PF often prove inadequate. This limitation has led to the exploration of AI approaches capable of uncovering nonlinear patterns in complex motion that traditional methods might miss due to their stochastic nature. AI can observe, interpret, and learn movement patterns, their interactions, and evolution, thus providing more accurate forecasts for complex and dynamic motions.

The Transformer model, introduced by Vaswani et al. \cite{transformer}, leverages the self-attention mechanism to enhance memory of past time steps, outperforming traditional recurrent neural networks (RNNs) such as LSTMs \cite{lstm} and LMUs \cite{lmu} in terms of accuracy and robustness. Transformers combine the advantages of convolutional networks (optimal feature extraction) and RNNs (memory) and are resilient against large, complex input data spaces. Examples of Transformer-based models include ChatGPT for natural language processing \cite{gpt} and TimeGPT for time series data processing \cite{timegpt}. However, a Transformer model specifically designed for forecasting human motion in sports, including multivariate and multi-task scenarios, is yet to be developed.
s is yet to be developed.