\section{Thesis Outline.}
\label{sec:outline}
This thesis is organized to systematically address the research questions and present the solution in a coherent manner. 

Chapter \ref{chapt:rel_work} reviews the existing literature, focusing on related work in human motion forecasting and other relevant fields. It distinguishes between model-based and data-driven approaches, establishing the groundwork for the methodologies employed in this~thesis.

Chapter \ref{chapt:theoretical_background} covers the theoretical background necessary to understand the methods used in this thesis. It delves into linear models, recurrent schemes, attention-based transformers, and newer architectures such as the BitNet, ensuring that readers have a comprehensive understanding of the technical~concepts.

Chapter \ref{chapt:method} details the methodological approach, including the idea, data pipeline, and specific model configurations for each type of architecture (linear, recurrent, and attention-based). This Chapter is crucial for understanding how the experiments are structured and why each model is~used.

Chapter \ref{chapt:experimental_setup} describes the experimental setup, including hardware and software configurations, datasets, and experiment design. It explains how different experiments are conducted to evaluate the models, making clear how each research question is~addressed.

Chapter \ref{chapt:results} presents the results of the experiments, analyzing the performance of each model according to the chosen metrics. It offers insights into the effectiveness of each approach and how they compare under different~settings.

Finally, Chapter \ref{chapt:conclusion} summarizes the findings, answers the research questions, and offers suggestions for future work, providing a comprehensive conclusion to the research presented in this~thesis.
