\section{Research Questions}
\label{sec:research_questions}
In light of the challenges outlined in the problem statement, the focus is to systematically explore the factors influencing the accuracy and effectiveness of human motion forecasting models in dynamic environments. To achieve the primary objective of identifying the most effective model, seven research questions guide the study.


\begin{itemize}
\item[\textbf{Q1}] \textbf{How do attention-based models compare to other models in general?} \\
In the context of human motion forecasting for team sports such as basketball and soccer, there is a growing interest in understanding the effectiveness of attention-based models, particularly Transformers. This question aims to investigate how the Transformer model performs in contrast to established models in this domain. Specifically, it is important to evaluate how the Transformer's performance differs from the baseline when using a combination of positional and velocity input information. Understanding this comparison is crucial for assessing the potential of the Transformer to outperform current state-of-the-art approaches in forecasting accuracy and robustness. This question will be answered in Section \ref{exp:init}.

    
\item[\textbf{Q2}] \textbf{How does model performance vary when positional or velocity data is excluded?} \\
In forecasting human motion, the choice of input data is crucial for achieving optimal model performance. This question examines how the performance of each model changes when either positional or velocity data is omitted. Excluding these data types can reduce computational complexity, making the models more efficient in practice. Therefore, it is important to assess whether such exclusions lead to a significant decline in accuracy and robustness. Understanding how models respond to different input configurations is essential for optimizing forecasting strategies and enhancing predictive capabilities in dynamic environments. This question will be answered in Section \ref{exp:pos_vel}.


    
\item[\textbf{Q3}] \textbf{How robust is each model to missing historical information?} \\
In forecasting human motion, the stability of models when faced with incomplete historical data is a critical concern. This investigation focuses on the performance of each model when historical information is missing or truncated. Certain models may inherently manage shorter time intervals more effectively, potentially making them better suited for tasks where extensive historical context is not available. Evaluating the robustness of these models under such conditions will provide insights into their reliability and adaptability in dynamic environments. Understanding how models cope with missing data is essential for identifying those that can maintain performance in challenging scenarios. This question will be answered in Section \ref{exp:history_forcast}.


\item[\textbf{Q4}] \textbf{How does a multivariate predictor compare to multiple univariate predictors in terms of forecasting accuracy?} \\
In the context of human motion forecasting, models can take two primary approaches: multivariate predictors, which forecast multiple variables simultaneously, and univariate predictors, which focus on one variable at a time. While multivariate models can effectively capture interactions between variables, they may also face challenges related to complexity and dependencies, potentially leading to inaccuracies. In contrast, univariate models simplify the prediction task but might overlook critical information from other variables. Given that many existing models in this field are multivariate, it is essential to compare these approaches regarding their forecasting accuracy. This analysis will clarify whether a unified multivariate strategy offers significant advantages or if individual univariate models produce better results for specific variables. This question will be answered in Section \ref{exp:uni_multi}.


\item[\textbf{Q5}] \textbf{How do the models generalize to other teams?} \\
In forecasting human motion, the ability of models to generalize across different team environments is essential. This analysis examines the outcomes of training a model on Team A and testing it on the same team versus training on Team A and testing on Team B. Assessing generalization is crucial for evaluating a model’s robustness and its applicability in real-world scenarios, where team dynamics and behaviors can vary significantly. By comparing performance in these situations, we can gain insights into each model’s adaptability and its potential for accurately forecasting motion across diverse contexts. This question will be addressed in Section \ref{exp:intra_inter}.


\item[\textbf{Q6}] \textbf{How does transfer learning impact model performance when trained on one domain and fine-tuned on another?} \\
In situations where data may be incomplete or unavailable, leveraging transfer learning becomes a critical strategy for enhancing model performance. Our aim is to create a model that is both generalizable and applicable to any team-based sport. Consequently, understanding how a pretrained model can be fine-tuned for specific sports is essential. This analysis explores the behavior of our models when initially trained on one domain and subsequently fine-tuned on another. Evaluating the effectiveness of transfer learning in this context is vital for optimizing human motion forecasting and ensuring adaptability across various sports environments. This question will be examined in Section \ref{exp:transf}.


\item[\textbf{Q7}] \textbf{How does each model perform when predicting the movement of a single target player using only that player's data, excluding the social-interaction context?} \\
In human motion forecasting, the choice of input data significantly impacts prediction accuracy. This analysis investigates how each model's performance varies when predicting the movement of a target player using only that player's data, thus excluding the social-interaction context. Understanding these differences is crucial for assessing whether relying solely on individual data affects predictive capabilities and the overall accuracy of the models.This question will be examined in Section \ref{exp:single_vs_all}.

\end{itemize}
